\question O que acontece com o tempo de execução do programa quando aumentamos o número de threads? Por que isso ocorre?
%\begin{parts}
    %\part Woah, subpart!
\begin{solution}
    Após um certo número de threads o tempo de execução do programa aumenta relativamente, isso se deve à necessidade de troca de contexto/overhead por conta da obrigação da sequencialização do trecho do código, como se fosse uma fila. Ao aumentar bastante o número dessas threads, essa fila aumenta cada vez mais, consequentemente aumentando o tempo de espera e seu tempo de execução.
\end{solution}
%\end{parts}